\documentclass[12pt]{report}
\usepackage[utf8]{inputenc}
\usepackage[russian]{babel}
%\usepackage[14pt]{extsizes}
\usepackage{listings}
\usepackage{graphicx}
\usepackage{amsmath,amsfonts,amssymb,amsthm,mathtools} 
\usepackage{pgfplots}
\usepackage{filecontents}
\usepackage{indentfirst}
\usepackage{eucal}
\usepackage{float} 
\usepackage{amsmath}
\usepackage{enumitem}
\usepackage[justification=centering]{caption} 
\usepackage{tikz}
\usepackage{pgfplots}
\pgfplotsset{compat=newest}

\frenchspacing

\usepackage{indentfirst} % Красная строка


%\usetikzlibrary{datavisualization}
%\usetikzlibrary{datavisualization.formats.functions}

\usepackage{amsmath}


% Для листинга кода:
\lstset{ %
language=haskell,                 % выбор языка для подсветки (здесь это С)
basicstyle=\small\sffamily, % размер и начертание шрифта для подсветки кода
numbers=left,               % где поставить нумерацию строк (слева\справа)
numberstyle=\tiny,           % размер шрифта для номеров строк
stepnumber=1,                   % размер шага между двумя номерами строк
numbersep=5pt,                % как далеко отстоят номера строк от подсвечиваемого кода
showspaces=false,            % показывать или нет пробелы специальными отступами
showstringspaces=false,      % показывать или нет пробелы в строках
showtabs=false,             % показывать или нет табуляцию в строках
frame=single,              % рисовать рамку вокруг кода
tabsize=2,                 % размер табуляции по умолчанию равен 2 пробелам
captionpos=t,              % позиция заголовка вверху [t] или внизу [b] 
breaklines=true,           % автоматически переносить строки (да\нет)
breakatwhitespace=false, % переносить строки только если есть пробел
escapeinside={\#*}{*)}   % если нужно добавить комментарии в коде
}

\usepackage[left=2cm,right=2cm, top=2cm,bottom=2cm,bindingoffset=0cm]{geometry}
% Для измененных титулов глав:
\usepackage{titlesec, blindtext, color} % подключаем нужные пакеты
\definecolor{gray75}{gray}{0.75} % определяем цвет
\newcommand{\hsp}{\hspace{20pt}} % длина линии в 20pt
% titleformat определяет стиль
\titleformat{\chapter}[hang]{\Huge\bfseries}{\thechapter\hsp\textcolor{gray75}{|}\hsp}{0pt}{\Huge\bfseries}


% plot
\usepackage{pgfplots}
\usepackage{filecontents}
\usetikzlibrary{datavisualization}
\usetikzlibrary{datavisualization.formats.functions}

\begin{document}
%\def\chaptername{} % убирает "Глава"
\thispagestyle{empty}
\begin{titlepage}
	\noindent \begin{minipage}{0.15\textwidth}
	\includegraphics[width=\linewidth]{report_files/bmstu_logo.jpg}
	\end{minipage}
	\noindent\begin{minipage}{0.9\textwidth}\centering
		\textbf{Министерство науки и высшего образования Российской Федерации}\\
		\textbf{Федеральное государственное бюджетное образовательное учреждение высшего образования}\\
		\textbf{~~~«Московский государственный технический университет имени Н.Э.~Баумана}\\
		\textbf{(национальный исследовательский университет)»}\\
		\textbf{(МГТУ им. Н.Э.~Баумана)}
	\end{minipage}
	
	\noindent\rule{18cm}{3pt}
	\newline\newline
	\noindent ФАКУЛЬТЕТ $\underline{\text{«Информатика и системы управления»}}$ \newline\newline
	\noindent КАФЕДРА $\underline{\text{«Программное обеспечение ЭВМ и информационные технологии»}}$\newline\newline\newline\newline\newline
	
	
	\begin{center}
		\noindent\begin{minipage}{1.3\textwidth}\centering
			\Large\textbf{  Отчёт по лабораторной работе №5}\newline
			\textbf{по дисциплине "Анализ алгоритмов"}\newline\newline
		\end{minipage}
	\end{center}
	
	\noindent\textbf{Тема} $\underline{\text{Конвейерные вычисления}}$\newline\newline
	\noindent\textbf{Студент} $\underline{\text{Костев Д.И.}}$\newline\newline
	\noindent\textbf{Группа} $\underline{\text{ИУ7-51Б}}$\newline\newline
	\noindent\textbf{Преподаватели} $\underline{\text{Волкова Л.Л.}}$\newline\newline\newline
	
	\begin{center}
		\vfill
		Москва~---~\the\year
		~г.
	\end{center}
\end{titlepage}

\setcounter{page}{2}
\tableofcontents

\newpage
\chapter*{Введение}
\addcontentsline{toc}{chapter}{Введение}

При обработке данных могут возникать ситуации, когда один набор данных необходимо обработать последовательно несколькими алгоритмами. В таком случае удобно использовать конвейерную обработку данных, что позволяет на каждой следующей <<линии>> конвейера использовать данные, полученные с предыдущего этапа. 

Помимо линейной конвейерной обработки данных, существуют параллельная конвейерная обработка данных. При таком подходе все линии работают с меньшим времени простоя, так как могут обрабатывать задачи независимо от других линий.

\section*{Цель лабораторной работы}

Целью данной лабораторной работы является изучение и реализация параллельной и линейной реализации конвейерной обработки данных.

\section*{Задачи лабораторной работы}

В рамках выполнения работы необходимо решить следующие задачи:

\begin{itemize}
	\item изучить конвейерную обработку данных;
	\item реализовать систему конвейерных вычислений с количеством линий не меньше трёх;
	\item сравнить параллельную и линейную реализацию конвейерных вычислений;
	\item сделать выводы на основе проделанной работы.
\end{itemize}

\chapter{Аналитическая часть}

\section{Конвейерная обработка данных}

Конвейер - система поточного производства. В терминах программирования ленты конвейера представлены функциями, выполняющими над неким набором данных операции и предающие их на следующую ленту конвейера. Моделирование конвейерной обработки хорошо сочетается с технологией многопоточного программирования - под каждую ленту конвейера выделяется отдельный поток, все потоки работают в асинхронном режиме.

\section{Описание задачи}

В качестве алгоритма, реализованного для распределения на конвейере, была использована аналогия с этапами сборки автомобилей. Сопоставление этапов приведено ниже. 

\begin{enumerate}
	\item Cборка двигателя (возведение числа в степень).
	\item Cборка корпуса (проверка числа на простоту).
	\item Cборка колёс (вычисление числа Фибоначчи).
\end{enumerate}

\section*{Вывод}
	В данном разделе были рассмотрены особенности построения конвейерных вычислений.

\chapter{Конструкторская часть}
\section{Разработка алгоритмов}

На рисунке 2.1 приведена схема организации конвейерных вычислений.

\begin{figure}[h]
	\centering
	\includegraphics[scale=0.6]{report_files/scheme_lab05.jpg}
	\caption{Схема организации конвейерных вычислений.}
	\label{fig:mpr}
\end{figure}


\section*{Вывод}
На основе теоретических данных, полученных аз аналитического раздела, были построенна схема алгоритма конвейерных вычислений.

\chapter{Технологическая часть}

В данном разделе приведены средства реализации и листинги кода.

\section{Требование к ПО}

К программе предъявляется ряд требований:

\begin{itemize}
	\item на вход подается количество задач (количество машин, которые нужно собрать)
	\item на выходе -- время, затраченное на обработку заявок;
	\item в процессе обрабатывания задач необходимо фиксировать время прихода и ухода заявки с линии.
\end{itemize}

\section{Средства реализации}

Для реализации ПО я выбрал язык программирования С++.

\section{Реализация алгоритмов}

В листингах 3.1 и 3.2 приведены реализации конвейерных вычислений (класс Cloveyor), реализация сборки машины (класс Car) и реализация класса отвечающего за логгирование (класс Logger).

\begin{lstlisting}[label=some-code,caption=Реализация класса конвейера, language=C++]
#include <thread>
#include <queue>

#include "car.h"

#define THRD_CNT 3

class Conveyor
{
public:
	Conveyor() = default;
	~Conveyor() = default;
	
	void run(size_t cars_cnt);
	
	void create_engine();
	void create_carcass();
	void create_wheels();
	
private:
	std::thread threads[THRD_CNT];
	std::vector<std::shared_ptr<Car>> cars;
	
	std::queue<std::shared_ptr<Car>> q1;
	std::queue<std::shared_ptr<Car>> q2;
	std::queue<std::shared_ptr<Car>> q3;
};

void Conveyor::run_parallel(size_t cars_cnt)
{
	for (size_t i = 0; i < cars_cnt; i++)
	{
		std::shared_ptr<Car> new_car(new Car);
		cars.push_back(new_car);
		q1.push(new_car);
	}
	
	this->threads[0] = std::thread(&Conveyor::create_carcass, this);
	this->threads[1] = std::thread(&Conveyor::create_engine, this);
	this->threads[2] = std::thread(&Conveyor::create_wheels, this);
	
	for (int i = 0; i < THRD_CNT; i++)
	{
		this->threads[i].join();
	}
}

void Conveyor::run_linear(size_t cars_cnt) 
{
    for (size_t i = 0; i < cars_cnt; i++)
    {
        std::shared_ptr<Car> new_car(new Car);
        cars.push_back(new_car);
        q1.push(new_car);
    }

    // Переменные для логгирования 
    tm start_1;
    size_t str1;
    tm stop_1;
    size_t stp1;
    
    tm start_2;
    size_t str2;
    tm stop_2;
    size_t stp2;

    tm start_3;
    size_t str3;
    tm stop_3;
    size_t stp3;


    for (size_t i = 0; i < cars_cnt; i++) 
    {
        std::shared_ptr<Car> car = q1.front();
        set_time(&start_1, &str1);

        car->create_carcass(i + 1);

        set_time(&stop_1, &stp1);
        q2.push(car);
        q1.pop();

        car = q2.front();
        set_time(&start_2, &str2);
        car->create_engine(i + 1);
        set_time(&stop_2, &stp2);


        q3.push(car);
        q2.pop();

        car = q3.front();
        set_time(&start_3, &str3);
        car->create_wheels(i + 1);
        set_time(&stop_3, &stp3);
        q3.pop();

        print_my(i + 1, 1, str1, start_1);
        print_my(i + 1, 1, stp1, stop_1);

        print_my(i + 1, 2, str2, start_1);
        print_my(i + 1, 2, stp2, stop_2);

        print_my(i + 1, 3, str3, start_1);
        print_my(i + 1, 3, stp3, stop_3);
    }
}

void Conveyor::create_carcass()
{
	size_t task_num = 0;
	
	while (!this->q1.empty())
	{
		std::shared_ptr<Car> car = q1.front();
		car->create_carcass(++task_num);
		
		q2.push(car);
		q1.pop();
	}
}

void Conveyor::create_engine()
{
	size_t task_num = 0;
	
	do
	{
		if (!this->q2.empty())
		{
			std::shared_ptr<Car> car = q2.front();
			car->create_engine(++task_num);
	
			q3.push(car);
			q2.pop();
		}
	} while(!q1.empty() || !q2.empty());
}

void Conveyor::create_wheels()
{
	size_t task_num = 0;
	
	do
	{
		if (!this->q3.empty())
		{
			std::shared_ptr<Car> car = q3.front();
			car->create_wheels(++task_num);
			q3.pop();
		}
	} while (!q1.empty() || !q2.empty() || !q3.empty());
}
\end{lstlisting}

\begin{lstlisting}[label=some-code,caption=Реализация класса сборки машины, language=C++]
#include <memory>
#include <cmath>

#include "logger.h"

class Carcass
{
public:
	Carcass(size_t num);
	~Carcass() = default;
	
	bool is_freight;
};

class Engine
{
public:
	Engine(int a, int x);
	~Engine() = default;
	
	size_t engine_power;
};

class Wheels
{
public:
	Wheels(int n);
	~Wheels() = default;
	
	size_t wheels_cnt;
};

class Car
{
public:
	Car() = default;
	~Car() = default;
	
	void create_engine(size_t);
	void create_carcass(size_t);
	void create_wheels(size_t);
	
private:
	std::unique_ptr<Carcass> carcass;
	std::unique_ptr<Engine> engine;
	std::unique_ptr<Wheels> wheels;
};

Carcass::Carcass(size_t num)
{
    this->is_freight = false;

    for (size_t i = 2; i <= sqrt(num); i++)
        if (0 == num % i)
            return;

    this->is_freight = true;
}

// Вычисление мощности движка (a^x)
Engine::Engine(int a, int x)
{    
    this->engine_power = a;

    for (int i = 0; i < x; i++)
        this->engine_power *= a;
}

// Вычисление числа колес (n-ое число Фибоначчи)
Wheels::Wheels(int n)
{
    size_t f1 = 1, f2 = 1;
    this->wheels_cnt = f1;

    for (int i = 2; i < n; i++)
    {
        this->wheels_cnt = f1 + f2;
        f1 = f2;
        f2 = this->wheels_cnt;
    }
}

void Car::create_engine(size_t task_num)
{
    if (this->carcass->is_freight) 
        this->engine = std::unique_ptr<Engine>(new Engine(10, 150000));
    else
        this->engine = std::unique_ptr<Engine>(new Engine(5,  150000));
}

void Car::create_carcass(size_t task_num)
{
    this->carcass = std::unique_ptr<Carcass>(new Carcass(27644437));
}

void Car::create_wheels(size_t task_num)
{
    this->wheels = std::unique_ptr<Wheels>(new Wheels(this->engine->engine_power));
}
}
\end{lstlisting}

\begin{lstlisting}[label=some-code,caption=Реализация класса логирования, language=C++]
#include <iostream>
#include <chrono>

using namespace std::chrono;

class Logger 
{
public:
	Logger() = default;
	~Logger() = default;
	
	static void set_time(tm *ti, size_t *ms)
}

void set_time(tm *ti, size_t *ms)
{
    system_clock::time_point now = system_clock::now();
    system_clock::duration cast = now.time_since_epoch();

    cast -= duration_cast<seconds>(cast);
    *ms = cast.count();
    
    time_t tt = system_clock::to_time_t(now);
    *ti = *gmtime(&tt);
}
\end{lstlisting}

%\section{Тестовые данные}

\section*{Вывод}

В данном разделе была разработана и рассмотрена реализация конвейерных вычислений.

\chapter{Исследовательская часть}

В данном разделе приведен анализ характеристик разработанного ПО  и примеры работы ПО.

\section{Пример работы программы}

На рисунке 4.1 приведна часть вывода программы.

\begin{figure}[hp!]
	\centering
	\includegraphics[scale=1.3]{report_files/app_run.jpg}
	\caption{Пример работы программы (параллельная обработка)}
	\label{fig:conveyor}
\end{figure}
\newpage
\section{Технические характеристики}

Ниже приведены технические характеристики устройства, на котором было проведено тестирование ПО:

\begin{itemize}
	\item Операционная система: Windows 11 64-bit.
	\item Оперативная память: 8 ГБ.
	\item Процессор: Intel(R) Core(TM) i5-8250 CPU @ 1.60 ГГц.
\end{itemize}

\section{Время выполнения реализаций алгоритмов}

Время выполнения алгоритма замерялось с помощью применения технологии профайлинга. Данный инструмент даёт детальное описание количество вызовов и количества времени CPU, затраченного на выполнение каждой функции. На вход функциям подаются обычные числа ввиду их простоты.
В таблице \ref{tab01} приведено сравнение времени выполнения параллельной обработки данных (сборка машины), в зависимости от количества входных задач (количества машин). Линия №1 - сборка каркасов автомобилей (проверка числа на простоту), линия №2 - сборка двигателей автомобилей (возведение числа в степень), линия №3 - сборка колёс автомобилей (вычисление числа Фибоначчи). Время указано в секундах.

\begin{table} [h!]
	\caption{Таблица времени выполнения параллельной обработки данных, время в секундах}
	\label{tab01}
	\begin{center}
		\begin{tabular}{|c c c c c|} 
		 	\hline
			Количество задач & Линия №1 & Линия №2 & Линия №3 & Общее время работы  \\  
		 	\hline
		 	50 & 0.03 & 0.16 & 0.01 & 0.27 \\
		 	\hline
		 	100 & 0.06 & 0.34 & 0.02 & 0.47 \\
		 	\hline
		 	200 & 0.13 & 0.63 & 0.06 & 0.9 \\
		 	\hline
			400 & 0.3 & 1.32 & 0.15 & 1.86 \\
			\hline
			800 & 0.63  & 2.45 & 0.31 & 3.45 \\
			\hline
		\end{tabular}
	\end{center}
\end{table}

\section*{Вывод}

В данном разделе приведено время исполнения линейной реализацией конвейера. Как видно из таблицы \ref{tab01}, вторая линия, то есть сборка двигателей (возведение числа в степень) занимает в среднем 70\% времени от выполнения всей программы. Линия №3 в среднем работает быстрее чем линия №1.

\chapter*{Заключение}
\addcontentsline{toc}{chapter}{Заключение}

В рамках данной лабораторной работы была достигнута её цель: изучена параллельная и линейная реализация конвейерной обработки данных. Также выполнены следующие задачи:

\begin{itemize}
	\item изучена конвейерная обработка данных
	\item реализована система конвейерных вычислений с количеством линий не меньше трёх;
	\item сравнены параллельные и линейные реализации конвейерных вычислений;
	\item сделаны выводы на основе проделанной работы;
\end{itemize}

Параллельные конвейерные вычисления позволяют организовать непрерывную обработку данных, что позволяет выиграть время в задачах, где требуется обработка больших объемов данных за малый промежуток времени.

% \addcontentsline{toc}{chapter}{Литература}

\bibliographystyle{utf8gost705u}  % стилевой файл для оформления по ГОСТу

\newpage
\addcontentsline{toc}{chapter}{Литература}
\begin{thebibliography}{}
\bibitem{bibStandart}
C++ Standard, Режим доступа: \url{https://isocpp.org/}, Дата обращения: 11.12.2021.
\bibitem{bibwiki}
Вычислительный конвейер, Режим доступа: \url{https://ru.wikipedia.org/wiki/Вычислительный\_конвейер}, Дата обращения: 11.12.2021.
\end{thebibliography}



\end{document}